\section{Introduction}  \label{sec:SNIPERIntro} 
Software-Defined Networking (SDN), along with its OpenFlow enabler, is an emerging technology that nicely separates the measurement data plane and control plane functions, and provides a capability to control and re-program the internal configurations of switches in dynamic environments. Such a flexibility can be used to adaptively and efficiently implement more complex networking applications, including both passive and active network monitoring without the need of customization. This is of particular importance in many network management and security applications where accurate and reliable network monitoring is necessary to provide critical information about internal characteristics or states of the network(s) that must be directly measured or indirectly inferred \cite{MYu:2011} \cite{MYu:2013} \cite{IF14iSTAMP:2014} \cite{Adrichen:2014}. 

In today's complex networks, the direct measurement of network's Internal Attributes of Interest (IAI) can be challenging or even inefficient and infeasible due to the hard constraints of network measurement resources. For example, consider the case where the IAI is the Origin-Destination Flow (ODF) sizes (as a measure of traffic intensity between nodes) or where the IAI are per-flow delay, throughput or packet loss. In large-scale networks, the measurement resources, including the Ternary Content Addressable Memory (TCAM) entries, processing power, storage capacity and limited available bandwidth, are very limited and providing per-flow measurements are not possible. Accordingly, Network Inference (NI) techniques, as powerful network monitoring tools, can be used to estimate the IAI based on a limited set of passive and/or active measurements. However, NI problems are naturally ill-posed in the sense that the number of measurements are not sufficient to uniquely and accurately determine the solution. Hence, side information from different perspectives and sources must be incorporated into the problem formulation to improve the estimation precision \cite{QZhao:2006} \cite{MDFE:2013} \cite{HNguyen:2007}. % \cite{Medina} \cite{Cao:2000} \cite{Roughan:2012}

The flexibility provided by the SDN can be utilized to optimize and facilitate the process of providing required direct measurements and/or side information which can then be used by network inference techniques to estimate the IAI. Accordingly, the capabilities of SDN have been utilized in a variety of passive and active network monitoring applications. For passive network measurement, the most SDN based works related to traffic engineering and network security applications where the main goal is focused on network traffic measurement or identifying Heavy Hitters (HH) and Hierarchical Heavy Hitters (HHH). In \cite{MYu:2011} and \cite{MYu:2013}, SDN reconfigurable measurement architectures are proposed where a variety of sketches for different direct measurement tasks can be defined and installed by the operator. In \cite{Tootoonchian:2010}, OpenTM directly measures a traffic matrix by keeping track of statistics for each flow. Recently, in \cite{IF14iSTAMP:2014}, an intelligent SDN based traffic measurement framework (called iSTAMP) with the ability of adaptive and accurate fine-grained flow estimation is proposed. For active network measurement under SDN paradigm, the very recent work \cite{Adrichen:2014} establishes a general framework (called Opennetmon) where accurate measurements of per-flow throughput, packet loss and delay can be directly measured.

% here for \cite{MYu:2013} IAI is a binary vector which represents the sate of measurements and inference is the HH detection technique (will be added to this paragraph)
However, under hard constraint of network measurement resources where network inference methods must be used to estimate the IAI from a set of limited measurements, the state of the art method in \cite{IF14iSTAMP:2014} suffers from the following challenges. First, the application of this framework is mainly limited to network traffic measurement. Second, the limited capability of current SDN infrastructures for providing required side information (due to longest prefix matching forwarding scheme in which incoming flows can be aggregated in just one entry of the TCAM) hardly limits the capability of providing optimal redundant aggregated measurements. Third, in \cite{IF14iSTAMP:2014}, to simplify the process of designing the optimal aggregation (i.e. measurement) matrix, the coherency of the measurement matrix is minimized which does not target the main estimation accuracy, and therefore, the unavoidable sacrifice in the performance is accepted \cite{IF14iSTAMP:2014}\cite{Elad:2007}.  
% Third, in \cite{IF14iSTAMP:2014}, the main objective function in the design of optimal aggregation matrix does not target the main estimation accuracy and, therefore, it accepts the unavoidable sacrifice in the performance.

On the other hand, recently, Matrix Completion (MC) techniques have been used as powerful network inference tools where the problem is the completion of a matrix of IAI from the direct measurement of a sub-set of its independent entries \cite{Roughan:2012}\cite{Gursun:2011}\cite{YLiao:2011}. Examples of the matrix of IAI include a matrix where each entry is an ODF at different times, or per-flow delay/packet-loss between different nodes of the network. Since in communication networks a variety of resources of the communication infrastructure at different layers are shared, the main assumption in MC techniques is that the matrix of IAI is a low-rank matrix which contains spatio-temporal redundancies and thus not all of its entries are needed to represent it; accordingly missed or non-observed entries can be estimated from a sub-set of randomly measured entries. In the theory of matrix completion, the matrix of IAI can be completely reconstructed from a sub-set of observed/measured entries (indicating the observation/measurement matrix) if the number of \emph{randomly} chosen observations are \emph{high} enough \cite{Candes:2009}\cite{Candes:2010}. Accordingly, in \cite{Roughan:2012} and \cite{Gursun:2011} the MC methods are used for network Traffic Matrix (TM) completion to estimate the missed entries of the TMs. Also, in \cite{YLiao:2011}, a new MC technique has been used for active network performance measurements where the status of path delays or bandwidths are predicted from a set of active measurements and using a new MC technique.

The construction of the matrix of IAI, where its entries are the measurements of \emph{independent} IAI, along with the flexibility provided by the SDN pave the way to: 1) design an efficient framework for different passive/active network measurement applications under hard constraint of measurement resources, and 2) provide required side information without feasibility constraints (as in \cite{IF14iSTAMP:2014}). Accordingly, in this paper, we use different matrix completion techniques as our main NI tools, and define an observation matrix as a matrix which provides required direct measurements of IAI that can be used by MC algorithms, and consequently, we answer the following interesting question:

\emph{Under hard resource constraints of network measurement resources, how can we optimally measure or sample a sub-set of entries of the matrix of IAI and design the optimal observation matrix which leads to the best possible estimation accuracy via using matrix completion techniques?}

However, the \emph{direct} design of optimal observation matrices for maximizing the performance of NI methods is prohibitive due to the complexity of the process \cite{IF14iSTAMP:2014}\cite{Elad:2007}. The underlying difficulty in this direct optimal observation matrix design, where the ultimate estimation accuracy of the NI process is targeted, is the fact that formulating the network inference process or algorithm into a closed-form and well-defined mathematical objective function that can be efficiently optimized is extremely complicated and computationally complex, if it is not impossible or intractable. Therefore, in this paper, we propose a new approach in designing the optimal observation matrix for network inference problems where we \emph{directly} target the ultimate estimation accuracy in network monitoring applications in our optimization framework. However, to cope with the inherent complexity of the process of designing large-scale optimal observation matrices, we use the well known Evolutionary Optimization Algorithms (EOA) that are suitable for the optimization problems where the main objective function is a procedure or an algorithm that can not be formulated as a well-defined mathematical function. In this framework, the evolutionary optimization algorithm acts as a \emph{sniper} which precisely captures or measures the best or the most informative entries of the matrix of IAI which leads to the best estimation accuracy via using matrix completion techniques. 

Under hard constraints of measurement resources in network monitoring applications, the SNIPER is a simple, flexible, and efficient framework which can be easily deployed on commodity OpenFlow-enabled routers/switches to enhance the performance of various passive or active network monitoring applications with low computation and communication overhead between control and data planes. Since MC techniques are the main NI methods in SNIPER, which directly use partial independent measurements of IAI, thus, this framework not only can be easily implemented in a centralized manner but also it can be implemented in distributed or decentralized ways. Accordingly, it is compatible with the recent trends in developing more smart and agile SDN platforms \cite{Bianchi:2014}\cite{Moshref:2014} where data plane APIs and switches are able to execute codes inside the device with no further interaction with the controller. 

Our main contributions are summarized as follows:

\textbf{$\text{\hspace{0.1cm}}\bullet$} To the best of our knowledge for the first time, we use the EOAs to design the optimal observation matrix where ultimate network inference performance is the main objective function to be optimized. We show that under hard constraint of measurement resources the optimal design of the observation matrix provides more accurate estimates in different network monitoring applications, including per-flow size and delay estimations.

\textbf{$\text{\hspace{0.1cm}}\bullet$} We address the scalability, deployability and feasibility of the SNIPER framework: a) by reducing the computational complexity of EOAs; b) by introducing a new algorithm for network measurement in dynamic environments, and c) by evaluating the performance of our framework using both synthetic and real network measurement traces from different network topologies in two main applications including network traffic and delay estimation, and further, by implementing a prototype of SNIPER in Mininet environment.

The rest of this paper is organized as follows. Section~\ref{sec:SNIPERSysDsc} provides an overview of the SNIPER framework and
the matrix completion techniques that we have used as our main NI methods.  In Section~\ref{sec:SNIPEREvlObsMtxDsg} we describes our
optimal observation matrix design procedure using the EOAs. Then, in Section~\ref{sec:SNIPERPerfEvalApp}, we explain our methodology for evaluating the performance of the SNIPER. Accordingly, in Section~\ref{sec:SNIPERResults}, we evaluate the performance of SNIPER considering two main applications including per-flow path delay and per-flow size estimations. Section~\ref{sec:Conclu} summarizes our most important results.
