\section{Introduction}  \label{sec:SNIPERIntro} 
Software-Defined Networking (SDN), along with its OpenFlow enabler, nicely separates the data plane and control plane functions, and provides a capability to control and re-program the internal configurations of switches in dynamic environments. Such a flexibility can be used to adaptively and efficiently implement more complex networking applications, including both passive and active network monitoring without the need of customization. This is of particular importance in many network management and security applications where accurate and reliable network monitoring is necessary to provide critical information about internal characteristics or states of the network(s) that must be directly measured or indirectly inferred \cite{MYu:2011} \cite{MYu:2013} \cite{IF14iSTAMP:2014} \cite{Adrichen:2014}. 

In today's complex networks, the direct measurement of network's Internal Attributes of Interest (IAI) can be challenging or even inefficient and infeasible due to the hard constraints of network measurement resources. For example, consider the case where the IAI is the Origin-Destination Flow (ODF) sizes (as a measure of traffic intensity between nodes) or where the IAI are per-flow delay, throughput or packet loss. In large-scale networks, the measurement resources, including the Ternary Content Addressable Memory (TCAM) entries, processing power, storage capacity and limited available bandwidth, are very limited, and hence, rendering per-flow measurements are infeasible. To cope with scalability issues, Network Inference (NI) techniques can be leveraged to estimate the IAI based on a limited set of passive and/or active measurements. However, NI problems are naturally ill-posed in the sense that the number of measurements are not sufficient to uniquely and accurately determine the solution. Hence, side information from different perspectives and sources must be incorporated into the problem formulation to improve the estimation precision \cite{QZhao:2006} \cite{MDFE:2013} \cite{HNguyen:2007}. % \cite{Medina} \cite{Cao:2000} \cite{Roughan:2012}

The flexibility provided by the SDN can be utilized to optimize and facilitate the process of collecting the required direct measurements and/or side information which can then be used by network inference techniques to estimate the IAI. In fact, the capabilities of SDN have been utilized in a variety of passive and active network monitoring applications. Most SDN based passive measurement studies are related to traffic engineering and network security applications, such as, network traffic measurement or identifying Heavy Hitters (HH) and Hierarchical Heavy Hitters (HHH). In \cite{MYu:2011} and \cite{MYu:2013}, SDN reconfigurable measurement architectures are proposed where a variety of sketches for different direct measurement tasks can be defined and installed by the operator. In \cite{Tootoonchian:2010}, OpenTM directly measures a traffic matrix by keeping track of statistics for each flow. Recently, in \cite{IF14iSTAMP:2014}, an intelligent SDN based traffic measurement framework (called iSTAMP) with the ability of adaptive and accurate fine-grained flow estimation is proposed. For active network measurement under SDN paradigm, the very recent work \cite{Adrichen:2014} establishes a general framework (called Opennetmon) where accurate measurements of per-flow throughput, packet loss and delay can be directly measured.

% here for \cite{MYu:2013} IAI is a binary vector which represents the sate of measurements and inference is the HH detection technique (will be added to this paragraph)
However, under hard resource constraints the state of the art SDN-enabled traffic measurement and inference methods, for example in \cite{MYu:2013} \cite{IF14iSTAMP:2014}, suffer from the following challenges. First, the application of these frameworks are mainly limited to network traffic measurement. Second, the longest prefix matching forwarding in OpenFlow implies that incoming flows can be aggregated in just one entry of the TCAM, and hence, the capability of providing optimal redundant aggregated measurements is limited. Third, in \cite{IF14iSTAMP:2014}, to simplify the process of designing the optimal aggregation (i.e. measurement) matrix, the ultimate estimation accuracy is not directly targeted. Instead, the coherency of the measurement matrix is minimized, leading to unavoidable sacrifice in the performance \cite{IF14iSTAMP:2014}\cite{Elad:2007}.  
% Third, in \cite{IF14iSTAMP:2014}, the main objective function in the design of optimal aggregation matrix does not target the main estimation accuracy and, therefore, it accepts the unavoidable sacrifice in the performance.

On the other hand, recently, Matrix Completion (MC) techniques have been used as powerful network inference tools that involve completing a matrix of IAI from the direct measurement of a sub-set of its independent entries \cite{Roughan:2012}\cite{Gursun:2011}\cite{YLiao:2011}. Examples of the matrix of IAI include a matrix where each entry is an ODF at different times \cite{Roughan:2012}, or per-flow delay/packet-loss between different nodes of the network \cite{YLiao:2011}. Since a variety of resources and information are often shared across different layers in communication networks, the main assumption in MC techniques is that the matrix of IAI is a low-rank matrix which contains spatio-temporal redundancies, and thus, not all of its entries are needed to represent it; accordingly missed or non-observed entries can be estimated from a sub-set of randomly measured entries. In the theory of matrix completion, the matrix of IAI can be completely reconstructed from a sub-set of observed/measured entries (indicating the observation/measurement matrix) if the number of \emph{randomly} chosen observations are \emph{high} enough \cite{Candes:2009}\cite{Candes:2010}. Accordingly, in \cite{Roughan:2012} and \cite{Gursun:2011} the MC methods are used for network Traffic Matrix (TM) completion to estimate the missed entries of the TMs. Also, in \cite{YLiao:2011}, a new MC technique has been used for active network performance measurements where the status of path delays or bandwidths are predicted from a set of active measurements and using a new MC technique.

The flexibility provided by the SDN coupled with the capability of MC techniques, in reconstructing the matrix of IAI from a sub-set of directly measured \emph{independent} entries, pave the way for: 1) designing an efficient framework for different passive/active network measurement applications under hard constraint of measurement resources, and 2) providing required side information without feasibility constraints (as in \cite{IF14iSTAMP:2014}). In this paper, we use different matrix completion techniques as our main NI tools, and define an observation matrix as a matrix which provides required direct measurements of IAI that can be used by MC algorithms. Specifically, we answer the following interesting question:

\emph{Under hard network resource constraints, how can we use the SDN capabilities to measure a sub-set of entries of the matrix of IAI and design the optimal observation matrix which leads to the best possible estimation accuracy using matrix completion techniques?}

However, the \emph{direct} design of optimal observation matrices for maximizing the performance of NI methods is prohibitive due to the complexity of the process \cite{IF14iSTAMP:2014}\cite{Elad:2007}. The underlying difficulty lies in the fact that formulating the network inference process or algorithm, as a function of the observation matrix which targets the ultimate estimation accuracy, into a closed-form and well-defined optimization problem that can be efficiently optimized is extremely complicated and computationally complex, if it is not impossible or intractable. Therefore, in this paper, we propose a new approach in designing the optimal observation matrix for network inference problems where we \emph{directly} target the ultimate estimation accuracy in network monitoring applications in our optimization framework. However, to cope with the inherent complexity of the process of designing large-scale optimal observation matrices, we use the well known Evolutionary Optimization Algorithms (EOA) that are suitable for the optimization problems where the main objective function is a procedure or an algorithm that can not be formulated as a well-defined mathematical function. In this framework, the evolutionary optimization algorithm acts as a \emph{sniper} which precisely captures or measures the best or the most informative entries of the matrix of IAI which leads to the best estimation accuracy via using matrix completion techniques. 
We refer to our proposed framework as Software defined Network Inference with Passive/active Evolutionary-optimal pRobing (SNIPER). The SNIPER is a simple, flexible, and efficient framework which can be easily deployed on commodity OpenFlow-enabled routers/switches to enhance the performance of various passive or active network monitoring applications with low computation and communication overhead between control and data planes. Since MC techniques (as the main NI methods employed by SNIPER) directly use partial independent measurements of IAI, this framework can be easily implemented in a centralized or distributed manner. Accordingly, it is compatible with the recent trends in developing more smart and agile SDN platforms \cite{Bianchi:2014}\cite{Moshref:2014} where data plane APIs and switches are able to execute codes inside the device with no further interaction with the controller. 

Our main contributions are summarized as follows:

\textbf{$\text{\hspace{0.1cm}}\bullet$} To the best of our knowledge, this is the first time that EOAs are applied to design the optimal observation matrix where ultimate network inference performance is the main objective function to be optimized. We show that under hard constraint of measurement resources the optimal design of the observation matrix provides more accurate estimates in different network monitoring applications, including per-flow size and delay estimations.

\textbf{$\text{\hspace{0.1cm}}\bullet$} We address the scalability, deployability and feasibility of the SNIPER framework: a) by reducing the computational complexity of EOAs; b) by introducing a new adaptive algorithm for network measurement in dynamic environments, and c) by evaluating the performance of our framework using both synthetic and real network measurement traces from different network topologies. We applied SNIPER to two main applications: network traffic and delay estimations. In addition, we implemented a prototype of SNIPER in Mininet.

The rest of this paper is organized as follows. Section~\ref{sec:SNIPERSysDsc} provides an overview of the SNIPER framework and
the matrix completion techniques that we have used as our main NI methods.  In Section~\ref{sec:SNIPEREvlObsMtxDsg} we describes our
optimal observation matrix design procedure using the EOAs. Then, in Section~\ref{sec:SNIPERPerfEvalApp}, we explain our methodology for evaluating the performance of the SNIPER. In Section~\ref{sec:SNIPERResults}, we evaluate the performance of SNIPER considering two main applications including per-flow path delay and per-flow size estimations. Section~\ref{sec:Conclu} summarizes our most important results.
