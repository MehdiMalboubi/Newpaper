\section{Introduction}  \label{sec:SNIPERIntro}
%MC needs independent measurement --> no feasible aggregability issues
%hard resource constraint
%mention to direct measurement and their problems 
%change intro and add related works section
In computer networks, network management refers to the activities, methods, procedures, and tools that pertain to the operation, administration, maintenance, provisioning and security of networked systems \cite{Clemmi:2006}. To meet the QoS agreements, a key requirement for network management is accurate and reliable network monitoring where critical information about internal characteristics or states of the network(s) must be directly measured or indirectly inferred. In today's complex networks, the direct measurement of network's Internal Attributes of Interest (IAI) can be challenging or even inefficient and infeasible due to the hard constraints of network measurement resources including the limited number of Ternary Content Addressable Memory (TCAM) entries at switches, the limited processing power and storage capacity and limited available bandwidth in active network performance measurement. Network Inference (NI) techniques are powerful network monitoring tools that can help estimate the IAI based on a limited set of measurements. Therefore, NI problems are naturally ill-posed in the sense that the number of measurements are not sufficient to uniquely and accurately determine the solution. Hence, side information from different perspectives and sources must be incorporated into the problem formulation to improve the estimation precision \cite{QZhao:2006} \cite{MDFE:2013} \cite{HNguyen:2007}. % \cite{Medina} \cite{Cao:2000} \cite{Roughan:2012}

Recently, Matrix Completion (MC) techniques have been used as network inference tools where the problem is the completion of a matrix of IAI from the direct measurement of a sub-set of its entries \cite{Roughan:2012}\cite{Gursun:2011}\cite{YLiao:2011}. Since in communication networks a variety of resources of the communication infrastructure at different layers are shared, the main assumption in MC techniques is that the matrix of IAI is a low-rank matrix which contains spatio-temporal redundancies and thus not all of its entries are needed to represent it; accordingly missed or non-observed entries can be estimated from a sub-set of randomly measured entries. In the theory of matrix completion, the matrix of IAI can be completely reconstructed from a sub-set of observed/measured entries (indicating the observation/measurement matrix) if the number of \emph{randomly} chosen observations are \emph{high} enough \cite{Candes:2009}\cite{Candes:2010}. Accordingly, in \cite{Roughan:2012} and \cite{Gursun:2011} the MC methods are used for network Traffic Matrix (TM) completion to estimate the missed entries of the TMs. Also, in \cite{YLiao:2011}, a new MC technique has been used for active network performance measurements where the status of path delays or bandwidths are predicted from a set of active measurements and using a new MC technique.

%On the other hand, nowadays, Software-Defined Networking (SDN) offers the required flexibility and interfaces to effectively implement different network monitoring, management and control tasks. In fact, an SDN enabler, such as OpenFlow, nicely separates the measurement data plane and control plane functions, and provides a capability to control/re-program the internal configurations of switches in dynamic environments. Consequently, SDN allows for more complex network monitoring and management applications \cite{IF14iSTAMP:2014} \cite{YLiao:2011} \cite{MYu:2011} \cite{MYu:2013}. %  in an adaptive and efficient way

On the other hand, Software-Defined Networking (SDN), along with its OpenFlow enabler, is an emerging technology that nicely separates the measurement data plane and control plane functions, and provides a capability to control/re-program the internal configurations of switches in dynamic environments. Consequently, SDN allows to adaptively and efficiently implement more complex network monitoring applications, including both passive and active network measurements, without the need of customization. For passive network measurement, the most SDN based works related to traffic engineering and network security applications where the main goal is focused on network traffic measurement or identifying Heavy Hitters (HH) and Hierarchical Heavy Hitters (HHH). In \cite{MYu:2011} and \cite{MYu:2013}, reconfigurable measurement architectures are proposed where a variety of sketches for different measurement tasks can be defined and installed by the operator. In \cite{Tootoonchian:2010}, OpenTM estimates a traffic matrix by keeping track of statistics for each flow. Recently, in \cite{IF14iSTAMP:2014}, an intelligent SDN based traffic measurement framework (called iSTAMP) with the ability of adaptive and accurate fine-grained flow estimation is proposed. For active network measurement under SDN paradigm, the very recent work \cite{Adrichen:2014} establishes a general framework where accurate measurements of per-flow throughput, packet loss and delay can be measured.

Here, since matrix completion techniques use the independent measurements of IAI, without suffering from the problem of feasibility of providing aggregated side information (such as the method of optimal flow aggregation in \cite{IF14iSTAMP:2014}), we use the flexibility provided by the SDN to address the following interesting question:

\emph{Under hard resource constraints of network measurement resources, how can we optimally measure or sample a sub-set of entries of the matrix of IAI and design the optimal observation matrix which leads to the best possible estimation accuracy via using matrix completion techniques?}
% and accordingly does not suffer from the problem of the feasibility of optimal aggregation techniques as in \cite{IF14iSTAMP:2014} for 
%Accordingly, in network monitoring applications, matrix completion techniques along with the flexibility provided by the SDN can be used to address the following interesting question:\emph{
%Under hard resource constraints of network measurement resources, how can we optimally measure or sample a sub-set of entries of the matrix of IAI and design the optimal observation matrix which leads to the best possible estimation accuracy via using matrix completion techniques?}
%% Such an optimal measurement/observation matrix is efficient in the sense that it needs less number of measurements to reach the same estimation accuracy via random measurement scheme, or equivalently, for the same number of measurements the optimal sampling techniques reaches a better perfromance.

However, the \emph{direct} design of optimal observation matrices for maximizing the performance of NI methods is prohibitive due to the complexity of the process \cite{IF14iSTAMP:2014}\cite{Elad:2007}. To simplify this process, other objective functions (e.g. coherency \cite{Elad:2007} or condition number \cite{MDFE:2013}) are considered in the optimization process, while accepting the unavoidable sacrifice in the performance \cite{Elad:2007}. The underlying difficulty in this direct optimal observation matrix design is that formulating the network inference process or algorithm into a closed-form and well-defined mathematical objective function that can be efficiently optimized is extremely complicated and computationally complex, if it is not impossible or intractable. 

Therefore, in this paper, we propose a new approach in designing the optimal observation matrix for network inference problems where we \emph{directly} target the ultimate estimation accuracy in network monitoring applications in our optimization framework. However, to cope with the inherent complexity of the process of designing large-scale optimal observation matrices, we use the well known Evolutionary Optimization Algorithms (EOA) that are suitable for the optimization problems where the main objective function is a procedure or an algorithm that can not be formulated as a well-defined mathematical function. In this framework, the evolutionary optimization algorithm acts as a \emph{sniper} which precisely captures or measures the best entries of the matrix of IAI which leads to the best estimation accuracy via matrix completion techniques. 
% The SNIPER framework is built upon the recent achievements in the theory of matrix completion to develop a practical foundation for the design of optimal measurement or observation matrices under hard resource constraints of network measurement resources.
%\subsection{Related Works}
%The flexibility provided by SDN and its enabler OpenFlow has been used to implement most of the passive and active network monitoring tasks in different applications without the need of customization. For passive network measurement, the most SDN based works related to traffic engineering and network security applications where the main goal is focused on network traffic measurement or identifying aspects of the network traffic such the presence of Heavy Hitters and Hierarchical heavy hitters. In \cite{MYu:2011}, the authors propose a reconfigurable measurement architecture for hierarchical heavy hitter detection, and then, \cite{MYu:2013} they propose a re-programmable structure (called OpenSketch) where a variety of sketches for different measurement tasks can be defined and installed by the operator. In \cite{Tootoonchian:2010}, OpenTM estimates a traffic matrix by keeping track of statistics for each flow. Recently, in \cite{IF14iSTAMP:2014} we have proposed an intelligent SDN based traffic measurement framework (called iSTAMP) with the ability of adaptive and accurate fine-grained flow estimation. For active network measurement under SDN paradigm, the very recent work \cite{Adrichen:2014} establishes a general framework where accurate measurements of flow throughput, packet loss and delay can be obtained.
%% \subsection{Our contributions}
%Under hard constraints of measurement resources in network monitoring applications, the SNIPER is a simple, generic, and efficient framework with the ability to optimally measure the most informative IAI which lead to the best achievable estimation accuracy via matrix completion methods. In fact, SNIPER can be easily deployed on commodity OpenFlow-enabled routers/switches to enhance the performance of various passive or active network monitoring applications with low computation and communication overhead between control and data planes. Here, we build upon the recent achievements in the theory of matrix completion to develop a practical foundation for guiding
%the design of optimal measurement or observation matrices under hard resource constraints of network measurement resources. 

Under hard constraints of measurement resources in network monitoring applications, the SNIPER is a simple, generic, and efficient framework which can be easily deployed on commodity OpenFlow-enabled routers/switches to enhance the performance of various passive or active network monitoring applications with low computation and communication overhead between control and data planes. Since MC techniques are the main NI methods in SNIPER, which directly use partial independent measurements of IAI, thus, this framework not only can be easily implemented in a centralized manner but also it can be implemented in distributed or decentralized ways. Accordingly, it is compatible with the recent trends in developing more smart and agile SDN platforms \cite{Bianchi:2014}\cite{Moshref:2014} where data plane APIs and switches are able to execute codes inside the device with no further interaction with the controller. Accordingly, our main contributions are summarized as follow:

\textbf{$\text{\hspace{0.1cm}}\bullet$} To the best of our knowledge for the first time, we use the EOAs to design the optimal observation matrix where ultimate network inference performance is the main objective function to be optimized, where we show that under hard constraint of measurement resources the optimal design of the observation matrix provides more accurate estimates.

\textbf{$\text{\hspace{0.1cm}}\bullet$} We address the scalability, deployability and feasibility of the SNIPER: a) by reducing the computational complexity of EOAs; b) by introducing a new online learning algorithm which concurrently measures and learns and c) by evaluating the performance of our framework using both synthetic and real network measurement traces from different network topologies in two main applications including network traffic and delay estimation, and further, by implementing a prototype of SNIPER in Mininet environment.

The rest of this paper is organized as follows. Section~\ref{sec:SNIPERSysDsc} provides an overview of SNIPER and
the matrix completion techniques that we have used as our main NI methods.  In Section~\ref{sec:SNIPEREvlObsMtxDsg} we describes our
optimal observation matrix design procedure using the EOAs. Then, in Section~\ref{sec:SNIPERPerfEvalApp}, we explain our methodology for evaluating the performance of the SNIPER. Accordingly, in Section~\ref{sec:SNIPERPerfEvalApp} we evaluate the performance of SNIPER considering two main applications including per-flow path delay and per-flow size estimations. Section~\ref{sec:Conclu} summarizes our most important results. 

%connect to resources allocation strategies (active learning)